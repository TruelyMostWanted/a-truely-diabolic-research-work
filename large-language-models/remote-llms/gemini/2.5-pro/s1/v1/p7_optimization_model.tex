\documentclass[11pt]{article}
\usepackage[a4paper, margin=1in]{geometry}
\usepackage{amsmath}
\usepackage{hyperref}

\title{Mathematical Optimization Model for Scrum Project Management}
\author{Generated by Gemini}
\date{\today}

\begin{document}

\maketitle
\tableofcontents
\newpage

\section{Sets (Entities)}
These sets are derived from the domain model entities and form the basis of the optimization problem. Each set contains all instances of a particular entity.

\begin{itemize}
    \item \textbf{Project (P)}: The set of all projects.
    \item \textbf{Team (T)}: The set of all teams.
    \item \textbf{Worker (W)}: The set of all workers.
    \item \textbf{Feature (F)}: The set of all features.
    \item \textbf{Skill (S)}: The set of all skills.
    \item \textbf{Role (R)}: The set of all Scrum roles.
    \item \textbf{ProductOwner (PO)}: The set of all product owners.
    \item \textbf{ScrumMaster (SM)}: The set of all scrum masters.
    \item \textbf{ProductBacklog (PB)}: The set of all product backlogs.
    \item \textbf{Sprint (SP)}: The set of all sprints.
    \item \textbf{UserStory (US)}: The set of all user stories.
    \item \textbf{Task (TSK)}: The set of all tasks.
    \item \textbf{Blocker (BL)}: The set of all blockers.
    \item \textbf{Stakeholder (SH)}: The set of all stakeholders.
    \item \textbf{ReleasePlan (REP)}: The set of all release plans.
\end{itemize}

\section{Indices}
These indices are used to iterate over the sets defined above in mathematical formulations.

\begin{itemize}
    \item $p \in P$: Index for a project.
    \item $t \in T$: Index for a team.
    \item $w \in W$: Index for a worker.
    \item $f \in F$: Index for a feature.
    \item $s \in S$: Index for a skill.
    \item $r \in R$: Index for a role.
    \item $po \in PO$: Index for a product owner.
    \item $sm \in SM$: Index for a scrum master.
    \item $pb \in PB$: Index for a product backlog.
    \item $sp \in SP$: Index for a sprint.
    \item $us \in US$: Index for a user story.
    \item $tsk \in TSK$: Index for a task.
    \item $bl \in BL$: Index for a blocker.
    \item $sh \in SH$: Index for a stakeholder.
    \item $rep \in REP$: Index for a release plan.
\end{itemize}

\section{Decision Variables}
These are the variables that the optimization model will determine. They represent the choices to be made, such as assignments and allocations.

\begin{itemize}
    \item \textbf{DV0: assign\_story\_to\_sprint} ($X_{us, sp} \in \{0,1\}$): 1 if User Story $us$ is assigned to Sprint $sp$, 0 otherwise.
    \item \textbf{DV1: assign\_worker\_to\_task} ($Y_{w, tsk} \in \{0,1\}$): 1 if Worker $w$ is assigned to Task $tsk$, 0 otherwise.
    \item \textbf{DV2: select\_feature\_for\_release} ($Z_{f, rep} \in \{0,1\}$): 1 if Feature $f$ is selected for Release Plan $rep$, 0 otherwise.
    \item \textbf{DV3: assign\_worker\_to\_team} ($A_{w, t} \in \{0,1\}$): 1 if Worker $w$ is assigned to Team $t$, 0 otherwise.
    \item \textbf{DV4: set\_team\_size} ($S_t \in \mathbb{Z}^+$): The number of workers assigned to Team $t$.
    \item \textbf{DV5: allocate\_worker\_availability} ($Avail_{w, sp} \in [0,1]$): The fraction of availability for Worker $w$ in Sprint $sp$.
    \item \textbf{DV10: choose\_blocker\_to\_resolve} ($B_{bl, sp} \in \{0,1\}$): 1 if Blocker $bl$ is chosen to be resolved in Sprint $sp$, 0 otherwise.
\end{itemize}

\section{Goals (Objective Function)}
The objective function defines the primary goals of the optimization. It is typically a weighted sum of various maximization and minimization goals. The final objective is to Maximize/Minimize a weighted sum of these terms.

\begin{itemize}
    \item \textbf{G0: maximize\_story\_points\_in\_sprint} \\
    Maximize: $1.0 \cdot \sum_{sp \in SP} \sum_{us \in US} \text{story\_points}_{us} \cdot X_{us, sp}$
    
    \item \textbf{G1: minimize\_effort\_for\_blockers} \\
    Minimize: $1.5 \cdot \sum_{sp \in SP} \sum_{bl \in BL} \text{effort}_{bl} \cdot B_{bl, sp}$
    
    \item \textbf{G2: maximize\_feature\_priority\_in\_release} \\
    Maximize: $1.2 \cdot \sum_{rep \in REP} \sum_{f \in F} \text{priority}_{f} \cdot Z_{f, rep}$
    
    \item \textbf{G6: maximize\_achievement\_of\_sprint\_goal} \\
    Maximize: $2.0 \cdot \sum_{sp \in SP} \text{achievement\_status}_{sp}$
    
    \item \textbf{G8: maximize\_stakeholder\_satisfaction} \\
    Maximize: $1.3 \cdot \sum_{sh \in SH} \sum_{f \in F} \text{relevance}_{sh, f} \cdot Z_{f, rep}$
\end{itemize}

\section{Conditions (Constraints)}
These are the rules and limitations that the solution must adhere to. They define the feasible region for the decision variables.

\begin{itemize}
    \item \textbf{C0: sprint\_effort\_must\_not\_exceed\_velocity} \\
    $\forall t \in T, \forall sp \in \text{SprintsOfTeam}(t): \sum_{us \in US} \text{story\_points}_{us} \cdot X_{us, sp} \le \text{velocity}_t$

    \item \textbf{C1: team\_size\_must\_be\_within\_scrum\_limits} \\
    $\forall t \in T: 3 \le S_t \le 9$, where $S_t = \sum_{w \in W} A_{w,t}$

    \item \textbf{C2: worker\_must\_not\_exceed\_availability} \\
    $\forall w \in W, \forall sp \in SP: \sum_{tsk \in TSK} \text{effort}_{tsk} \cdot Y_{w, tsk} \le Avail_{w, sp} \cdot \text{SprintDuration}_{sp}$

    \item \textbf{C4: user\_story\_must\_have\_criteria} \\
    $\forall us \in US, \forall sp \in SP: X_{us, sp} \le \text{HasAcceptanceCriteria}_{us}$, where $\text{HasAcceptanceCriteria}_{us}$ is 1 if true, 0 if false.

    \item \textbf{C5: team\_must\_have\_scrum\_master} \\
    $\forall t \in T: \exists sm \in SM$ such that relationship $R6(\text{Team}_t, \text{ScrumMaster}_{sm})$ is true.
    
    \item \textbf{C6: task\_cannot\_be\_done\_if\_blocked} \\
    $\forall tsk \in TSK: \text{status}_{tsk} \neq \text{'done'} \text{ if } \exists bl \in BL$ with $R16(\text{Task}_{tsk}, \text{Blocker}_{bl})$ and $\text{status}_{bl} \neq \text{'resolved'}$.

    \item \textbf{C7: prefer\_certified\_skills (Soft Constraint)} \\
    This can be added to the objective function as a maximization term: Maximize $\sum_{w \in W} \sum_{tsk \in TSK} \text{HasSkill}_{w, \text{req(tsk)}} \cdot Y_{w, tsk}$, where $\text{HasSkill}$ is a parameter indicating if worker $w$ has the skill required for task $tsk$.

    \item \textbf{C9: project\_must\_adhere\_to\_end\_date} \\
    $\forall sp \in SP: \text{end\_date}_{sp} \le \text{project\_end}_{\text{project(sp)}}$
\end{itemize}

\end{document}