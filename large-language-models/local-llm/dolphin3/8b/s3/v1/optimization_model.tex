\documentclass{article}
\title{Optimization Model for Software Development Company}
\author{Dolphin}
\date{\today}
\begin{document}
\maketitle
\tableofcontents

\section*{1. Sets (Entities)}
\begin{itemize}
    \item Project: A product or initiative to be developed.
    \item Team: Self-organized, cross-functional development team.
    \item Employee: Individual team member working on the project.
    \item Feature: Mid-sized functionality.
    \item Skill: Professional or social competence of a worker.
    \item Role: Defined responsibilities within the Scrum team.
    \item Product Owner: Responsible for product vision and Product Backlog.
    \item Scrum Master: Supports the team in applying Scrum.
    \item Product Backlog: Ordered list of all requirements.
    \item Sprint: Fixed time period for creating an increment.
    \item Sprint Planning: Kick-off meeting for Sprint preparation.
    \item Daily Scrum: Daily 15-minute team meeting.
    \item Sprint Review: Presentation and acceptance of results.
    \item Sprint Retrospective: Retrospective for process improvement.
    \item Sprint Backlog: Selected backlog items + implementation plan.
    \item Sprint Goal: Objective to be achieved within the sprint.
    \item Epic: Large requirement that can be split into stories.
    \item User Story: Requirement from the perspective of a user.
    \item Task: Smallest unit of work within a sprint.
    \item Development Snapshot: Product at the end of a sprint.
    \item Blocker: Obstacle hindering progress.
    \item Stakeholder: Interested party in the product (internal/external).
    \item Velocity: Average amount of work per sprint.
    \item Release Plan: Plan for releasing specific features.
    \item Roadmap: Long-term planning across releases.
    \item Scrum Board: Visual representation of tasks during the sprint.
    \item Feature Documentation: Documentation for a specific feature.
\end{itemize}

\section*{2. Indices}
\begin{itemize}
    \item ID: Unique identifier assigned to each entity.
    \item Name: The name given to an entity.
    \item SetName: Indicates the set or category that an entity belongs to.
    \item Index: A unique index value for each attribute.
    \item Description: Detailed description of an entity or its attributes.
    \item Attribute0-Attribute8: Specific attributes associated with an entity, depending on the type of entity.
\end{itemize}

\section*{3. Goals}
\begin{itemize}
    \item ID: Unique identifier assigned to each goal.
    \item Name: The name given to a goal.
    \item Description: Detailed description of a goal.
    \item IsSum: Boolean value indicating whether the goal is a sum-type variable or not.
    \item GoalType: Type of optimization goal, either "max" (maximum) or "min" (minimum).
    \item EntityName: Name of the entity targeted by the goal.
    \item EntityAttribute: Attribute of the targeted entity.
    \item CriteriaType: Number representing the criteria type for matching conditions with goals.
    \item Weight: Multiplier for the goal.
\end{itemize}

\section*{4. Conditions}
\begin{itemize}
    \item ID: Unique identifier assigned to each condition.
    \item Name: The name given to a condition.
    \item Description: Detailed description of a condition.
    \item IsSum: Boolean value indicating whether the condition is a sum-type variable or not.
    \item GoalType: Type of optimization goal, either "max" (maximum) or "min" (minimum).
    \item EntityName: Name of the entity targeted by the condition.
    \item EntityAttribute: Attribute of the targeted entity.
    \item CriteriaType: Number representing the criteria type for matching conditions with goals.
    \item Weight: Multiplier for the condition.
\end{itemize}

\section*{5. DecisionVariables}
\begin{itemize}
    \item ID: Unique identifier assigned to each decision variable.
    \item Name: The name given to a decision variable.
    \item Description: Detailed description of a decision variable.
    \item DataType: Type of data representing the decision variable, such as continuous or discrete.
    \item Domain: Set of values that the decision variable can take on.
    \item MinValue: Smallest value that the decision variable can take.
    \item MaxValue: Highest value that the decision variable can take.
\end{itemize}

\end{document}