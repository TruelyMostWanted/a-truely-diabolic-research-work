\documentclass{article}
\usepackage[utf8]{inputenc}
\usepackage{amsmath}

\title{Optimization Model Formulation}
\author{Your Name}
\date{\today}

\begin{document}

\maketitle

\tableofcontents

\section{Sets (Entities)}
These are the entities involved in the optimization model. Each entity is represented as a set.
\begin{itemize}
    \item Entity 1
    \item Entity 2
    \item Entity 3
    % Add more entities as needed
\end{itemize}

\section{Indices}
These are the indices used to represent elements within sets.
\begin{itemize}
    \item Index 1: Represents element in Set A
    \item Index 2: Represents element in Set B
    \item Index 3: Represents element in Set C
    % Add more indices as needed
\end{itemize}

\section{Goals}
These are the objectives of the optimization model.
\begin{itemize}
    \item Goal 1: 
        \begin{description}
            \item[Name:] Minimize Cost
            \item[Mathematical Representation:] $\min \sum_{i} c_i x_i$
        \end{description}
    \item Goal 2:
        \begin{description}
            \item[Name:] Maximize Profit
            \item[Mathematical Representation:] $\max \sum_{j} p_j y_j$
        \end{description}
    % Add more goals as needed
\end{itemize}

\section{Conditions}
These are the constraints of the optimization model.
\begin{itemize}
    \item Condition 1:
        \begin{description}
            \item[Name:] Resource Constraint
            \item[Mathematical Representation:] $\sum_{i} a_i x_i \leq b$
        \end{description}
    \item Condition 2:
        \begin{description}
            \item[Name:] Demand Constraint
            \item[Mathematical Representation:] $\sum_{j} d_j y_j \geq e$
        \end{description}
    % Add more conditions as needed
\end{itemize}

\section{Decision Variables}
These are the variables used in the optimization model.
\begin{itemize}
    \item Decision Variable 1: $x_i$ represents quantity of product i
    \item Decision Variable 2: $y_j$ represents allocation to resource j
    \item Decision Variable 3: $z_k$ represents binary decision for option k
    % Add more decision variables as needed
\end{itemize}

\end{document}